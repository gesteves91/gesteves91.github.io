
\documentclass{beamer}

%\usetheme{AnnArbor}
%\usetheme{Antibes}
%\usetheme{Bergen}
%\usetheme{Berkeley}
%\usetheme{Berlin}
%\usetheme{Boadilla}
%\usetheme{boxes}
%\usetheme{CambridgeUS}
%\usetheme{Copenhagen}
%\usetheme{Darmstadt}
%\usetheme{default}
%\usetheme{Frankfurt}
%\usetheme{Goettingen}
%\usetheme{Hannover}
%\usetheme{Ilmenau}
%\usetheme{JuanLesPins}
%\usetheme{Luebeck}
\usetheme{Madrid}
%\usetheme{Malmoe}
%\usetheme{Marburg}
%\usetheme{Montpellier}
%\usetheme{PaloAlto}
%\usetheme{Pittsburgh}
%\usetheme{Rochester}
%\usetheme{Singapore}
%\usetheme{Szeged}
%\usetheme{Warsaw}

%\usecolortheme{beaver}
%\usecolortheme{wolverine}
%\usecolortheme{dove}
%\usecolortheme{albatross}
%\usecolortheme{dolphin}
%\usecolortheme{orchid}
\usecolortheme{seahorse}
%\usecolortheme{beetle}
%\usecolortheme{whale}
%\usecolortheme{crane}
%\usecolortheme{lily}
%\setbeamercolor*{item}{fg=red}

% Include important packages
\usepackage[brazil]{babel}
\usepackage[utf8]{inputenc}
\usepackage{color}
\usepackage{tikz}

% Início
\title{Introdução à Computação}

\author{Geanderson Esteves dos Santos}

\institute[] 
{
  Pontifícia Universidade Católica de Minas Gerais\\
  Instituto de Ciências Exatas e Informática
}

\date{IC (2018/02)}

\subject{Theoretical Computer Science}

\AtBeginSubsection[]
{
  \begin{frame}<beamer>{Topicos}
    \tableofcontents[currentsection,currentsubsection]
  \end{frame}
}

% Let's get started
\begin{document}

\begin{frame}
  \titlepage
\end{frame}

% Slide 3
\begin{frame}{Apresentação da Disciplina}
Desenvolver as seguintes habilidades:
\begin{itemize}
\item Estrutura e funcionamento básicos de computadores.
\item Conceitos básicos de Sistemas Operacionais, Redes, Internet e Web.
\item Criação, implementação e manutenção de páginas na Web
\item Dispositivos de armazenamento de dados.
\item Principais áreas da Computação e suas aplicações nas organizações.
\end{itemize}
\end{frame}

% Slide 4
\begin{frame}{Professor - Geanderson Esteves dos Santos}
5 anos de experiência no mercado de TI trabalhando em empresas como Google, Avenue Code, CEMIG, e BHS. Atuando principalmente no desenvolvimento de software.
\begin{itemize}
\item Formação
\begin{itemize}
    \item Doutorando em Ciências da Computação - UFMG (Presente)
    \item Mestrado em Ciências da Computação - UFMG (2018)
    \item Intercâmbio em Ciências da Computação - Indiana University (2014)
    \item Bacharel em Sistemas de Informação - PUC Minas (2015)
\end{itemize}
\item Atuação
\begin{itemize}
    \item Machine Learning
    \item Análise de Sentimento
    \item Engenharia de Software
    \item Interação Humano-Computador
    \item Jogos para incentivar o aprendizado
\end{itemize}
\item Certificações
\begin{itemize}
    \item Microsoft Certifications in Web development and Databases
\end{itemize}
\end{itemize}
\end{frame}

% Slide 5
\usebackgroundtemplate{%             declare it
\begin{tikzpicture}[overlay, remember picture]
\node[anchor=north west, %anchor is upper left corner of the graphic
      xshift=7cm, %shifting around
      yshift=-2.5cm] 
     at (current page.north west) %left upper corner of the page
     {\includegraphics[width=4.5cm]{tests.png}}; 
\end{tikzpicture}}

\begin{frame}{Apresentação da Disciplina - Avaliação}
\begin{itemize}
\item Lista de exercícios: 20 pontos
\begin{itemize}
    \item Listas Variadas - 10 pontos
\end{itemize}
\item Provas: 60 pontos
\begin{itemize}
    \item Prova 1 - 30 pontos
    \item Prova 2 - 30 pontos
\end{itemize}
\item Trabalho Final - 30 pontos
\end{itemize}
\end{frame}

\usebackgroundtemplate{ }


\begin{frame}[allowframebreaks]
  \frametitle<presentation>{Apresentação da Disciplina – Referências Bibliográficas}
    
  \begin{thebibliography}{10}
    
  \beamertemplatebookbibitems
  % Start with overview books.

  \bibitem{Author1990}
    A.~Author.
    \newblock {\em Handbook of Everything}.
    \newblock Some Press, 1990.
 
    
  \beamertemplatearticlebibitems
  % Followed by interesting articles. Keep the list short. 

  \bibitem{Someone2000}
    S.~Someone.
    \newblock On this and that.
    \newblock {\em Journal of This and That}, 2(1):50--100,
    2000.
  \end{thebibliography}
\end{frame}

\end{document}


